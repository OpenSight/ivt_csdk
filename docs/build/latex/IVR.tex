% Generated by Sphinx.
\def\sphinxdocclass{report}
\documentclass[letterpaper,10pt,english]{sphinxmanual}
\usepackage[utf8]{inputenc}
\DeclareUnicodeCharacter{00A0}{\nobreakspace}
\usepackage{cmap}
\usepackage[T1]{fontenc}

\usepackage{babel}
\usepackage{times}
\usepackage[Bjarne]{fncychap}
\usepackage{longtable}
\usepackage{sphinx}
\usepackage{multirow}
\usepackage{eqparbox}


\addto\captionsenglish{\renewcommand{\figurename}{Fig. }}
\addto\captionsenglish{\renewcommand{\tablename}{Table }}
\SetupFloatingEnvironment{literal-block}{name=Listing }



\title{IVR Documentation}
\date{March 02, 2016}
\release{0.0.1}
\author{OpenSight}
\newcommand{\sphinxlogo}{}
\renewcommand{\releasename}{Release}
\setcounter{tocdepth}{1}
\makeindex

\makeatletter
\def\PYG@reset{\let\PYG@it=\relax \let\PYG@bf=\relax%
    \let\PYG@ul=\relax \let\PYG@tc=\relax%
    \let\PYG@bc=\relax \let\PYG@ff=\relax}
\def\PYG@tok#1{\csname PYG@tok@#1\endcsname}
\def\PYG@toks#1+{\ifx\relax#1\empty\else%
    \PYG@tok{#1}\expandafter\PYG@toks\fi}
\def\PYG@do#1{\PYG@bc{\PYG@tc{\PYG@ul{%
    \PYG@it{\PYG@bf{\PYG@ff{#1}}}}}}}
\def\PYG#1#2{\PYG@reset\PYG@toks#1+\relax+\PYG@do{#2}}

\expandafter\def\csname PYG@tok@gd\endcsname{\def\PYG@tc##1{\textcolor[rgb]{0.63,0.00,0.00}{##1}}}
\expandafter\def\csname PYG@tok@gu\endcsname{\let\PYG@bf=\textbf\def\PYG@tc##1{\textcolor[rgb]{0.50,0.00,0.50}{##1}}}
\expandafter\def\csname PYG@tok@gt\endcsname{\def\PYG@tc##1{\textcolor[rgb]{0.00,0.27,0.87}{##1}}}
\expandafter\def\csname PYG@tok@gs\endcsname{\let\PYG@bf=\textbf}
\expandafter\def\csname PYG@tok@gr\endcsname{\def\PYG@tc##1{\textcolor[rgb]{1.00,0.00,0.00}{##1}}}
\expandafter\def\csname PYG@tok@cm\endcsname{\let\PYG@it=\textit\def\PYG@tc##1{\textcolor[rgb]{0.25,0.50,0.56}{##1}}}
\expandafter\def\csname PYG@tok@vg\endcsname{\def\PYG@tc##1{\textcolor[rgb]{0.73,0.38,0.84}{##1}}}
\expandafter\def\csname PYG@tok@m\endcsname{\def\PYG@tc##1{\textcolor[rgb]{0.13,0.50,0.31}{##1}}}
\expandafter\def\csname PYG@tok@mh\endcsname{\def\PYG@tc##1{\textcolor[rgb]{0.13,0.50,0.31}{##1}}}
\expandafter\def\csname PYG@tok@cs\endcsname{\def\PYG@tc##1{\textcolor[rgb]{0.25,0.50,0.56}{##1}}\def\PYG@bc##1{\setlength{\fboxsep}{0pt}\colorbox[rgb]{1.00,0.94,0.94}{\strut ##1}}}
\expandafter\def\csname PYG@tok@ge\endcsname{\let\PYG@it=\textit}
\expandafter\def\csname PYG@tok@vc\endcsname{\def\PYG@tc##1{\textcolor[rgb]{0.73,0.38,0.84}{##1}}}
\expandafter\def\csname PYG@tok@il\endcsname{\def\PYG@tc##1{\textcolor[rgb]{0.13,0.50,0.31}{##1}}}
\expandafter\def\csname PYG@tok@go\endcsname{\def\PYG@tc##1{\textcolor[rgb]{0.20,0.20,0.20}{##1}}}
\expandafter\def\csname PYG@tok@cp\endcsname{\def\PYG@tc##1{\textcolor[rgb]{0.00,0.44,0.13}{##1}}}
\expandafter\def\csname PYG@tok@gi\endcsname{\def\PYG@tc##1{\textcolor[rgb]{0.00,0.63,0.00}{##1}}}
\expandafter\def\csname PYG@tok@gh\endcsname{\let\PYG@bf=\textbf\def\PYG@tc##1{\textcolor[rgb]{0.00,0.00,0.50}{##1}}}
\expandafter\def\csname PYG@tok@ni\endcsname{\let\PYG@bf=\textbf\def\PYG@tc##1{\textcolor[rgb]{0.84,0.33,0.22}{##1}}}
\expandafter\def\csname PYG@tok@nl\endcsname{\let\PYG@bf=\textbf\def\PYG@tc##1{\textcolor[rgb]{0.00,0.13,0.44}{##1}}}
\expandafter\def\csname PYG@tok@nn\endcsname{\let\PYG@bf=\textbf\def\PYG@tc##1{\textcolor[rgb]{0.05,0.52,0.71}{##1}}}
\expandafter\def\csname PYG@tok@no\endcsname{\def\PYG@tc##1{\textcolor[rgb]{0.38,0.68,0.84}{##1}}}
\expandafter\def\csname PYG@tok@na\endcsname{\def\PYG@tc##1{\textcolor[rgb]{0.25,0.44,0.63}{##1}}}
\expandafter\def\csname PYG@tok@nb\endcsname{\def\PYG@tc##1{\textcolor[rgb]{0.00,0.44,0.13}{##1}}}
\expandafter\def\csname PYG@tok@nc\endcsname{\let\PYG@bf=\textbf\def\PYG@tc##1{\textcolor[rgb]{0.05,0.52,0.71}{##1}}}
\expandafter\def\csname PYG@tok@nd\endcsname{\let\PYG@bf=\textbf\def\PYG@tc##1{\textcolor[rgb]{0.33,0.33,0.33}{##1}}}
\expandafter\def\csname PYG@tok@ne\endcsname{\def\PYG@tc##1{\textcolor[rgb]{0.00,0.44,0.13}{##1}}}
\expandafter\def\csname PYG@tok@nf\endcsname{\def\PYG@tc##1{\textcolor[rgb]{0.02,0.16,0.49}{##1}}}
\expandafter\def\csname PYG@tok@si\endcsname{\let\PYG@it=\textit\def\PYG@tc##1{\textcolor[rgb]{0.44,0.63,0.82}{##1}}}
\expandafter\def\csname PYG@tok@s2\endcsname{\def\PYG@tc##1{\textcolor[rgb]{0.25,0.44,0.63}{##1}}}
\expandafter\def\csname PYG@tok@vi\endcsname{\def\PYG@tc##1{\textcolor[rgb]{0.73,0.38,0.84}{##1}}}
\expandafter\def\csname PYG@tok@nt\endcsname{\let\PYG@bf=\textbf\def\PYG@tc##1{\textcolor[rgb]{0.02,0.16,0.45}{##1}}}
\expandafter\def\csname PYG@tok@nv\endcsname{\def\PYG@tc##1{\textcolor[rgb]{0.73,0.38,0.84}{##1}}}
\expandafter\def\csname PYG@tok@s1\endcsname{\def\PYG@tc##1{\textcolor[rgb]{0.25,0.44,0.63}{##1}}}
\expandafter\def\csname PYG@tok@gp\endcsname{\let\PYG@bf=\textbf\def\PYG@tc##1{\textcolor[rgb]{0.78,0.36,0.04}{##1}}}
\expandafter\def\csname PYG@tok@sh\endcsname{\def\PYG@tc##1{\textcolor[rgb]{0.25,0.44,0.63}{##1}}}
\expandafter\def\csname PYG@tok@ow\endcsname{\let\PYG@bf=\textbf\def\PYG@tc##1{\textcolor[rgb]{0.00,0.44,0.13}{##1}}}
\expandafter\def\csname PYG@tok@sx\endcsname{\def\PYG@tc##1{\textcolor[rgb]{0.78,0.36,0.04}{##1}}}
\expandafter\def\csname PYG@tok@bp\endcsname{\def\PYG@tc##1{\textcolor[rgb]{0.00,0.44,0.13}{##1}}}
\expandafter\def\csname PYG@tok@c1\endcsname{\let\PYG@it=\textit\def\PYG@tc##1{\textcolor[rgb]{0.25,0.50,0.56}{##1}}}
\expandafter\def\csname PYG@tok@kc\endcsname{\let\PYG@bf=\textbf\def\PYG@tc##1{\textcolor[rgb]{0.00,0.44,0.13}{##1}}}
\expandafter\def\csname PYG@tok@c\endcsname{\let\PYG@it=\textit\def\PYG@tc##1{\textcolor[rgb]{0.25,0.50,0.56}{##1}}}
\expandafter\def\csname PYG@tok@mf\endcsname{\def\PYG@tc##1{\textcolor[rgb]{0.13,0.50,0.31}{##1}}}
\expandafter\def\csname PYG@tok@err\endcsname{\def\PYG@bc##1{\setlength{\fboxsep}{0pt}\fcolorbox[rgb]{1.00,0.00,0.00}{1,1,1}{\strut ##1}}}
\expandafter\def\csname PYG@tok@mb\endcsname{\def\PYG@tc##1{\textcolor[rgb]{0.13,0.50,0.31}{##1}}}
\expandafter\def\csname PYG@tok@ss\endcsname{\def\PYG@tc##1{\textcolor[rgb]{0.32,0.47,0.09}{##1}}}
\expandafter\def\csname PYG@tok@sr\endcsname{\def\PYG@tc##1{\textcolor[rgb]{0.14,0.33,0.53}{##1}}}
\expandafter\def\csname PYG@tok@mo\endcsname{\def\PYG@tc##1{\textcolor[rgb]{0.13,0.50,0.31}{##1}}}
\expandafter\def\csname PYG@tok@kd\endcsname{\let\PYG@bf=\textbf\def\PYG@tc##1{\textcolor[rgb]{0.00,0.44,0.13}{##1}}}
\expandafter\def\csname PYG@tok@mi\endcsname{\def\PYG@tc##1{\textcolor[rgb]{0.13,0.50,0.31}{##1}}}
\expandafter\def\csname PYG@tok@kn\endcsname{\let\PYG@bf=\textbf\def\PYG@tc##1{\textcolor[rgb]{0.00,0.44,0.13}{##1}}}
\expandafter\def\csname PYG@tok@o\endcsname{\def\PYG@tc##1{\textcolor[rgb]{0.40,0.40,0.40}{##1}}}
\expandafter\def\csname PYG@tok@kr\endcsname{\let\PYG@bf=\textbf\def\PYG@tc##1{\textcolor[rgb]{0.00,0.44,0.13}{##1}}}
\expandafter\def\csname PYG@tok@s\endcsname{\def\PYG@tc##1{\textcolor[rgb]{0.25,0.44,0.63}{##1}}}
\expandafter\def\csname PYG@tok@kp\endcsname{\def\PYG@tc##1{\textcolor[rgb]{0.00,0.44,0.13}{##1}}}
\expandafter\def\csname PYG@tok@w\endcsname{\def\PYG@tc##1{\textcolor[rgb]{0.73,0.73,0.73}{##1}}}
\expandafter\def\csname PYG@tok@kt\endcsname{\def\PYG@tc##1{\textcolor[rgb]{0.56,0.13,0.00}{##1}}}
\expandafter\def\csname PYG@tok@sc\endcsname{\def\PYG@tc##1{\textcolor[rgb]{0.25,0.44,0.63}{##1}}}
\expandafter\def\csname PYG@tok@sb\endcsname{\def\PYG@tc##1{\textcolor[rgb]{0.25,0.44,0.63}{##1}}}
\expandafter\def\csname PYG@tok@k\endcsname{\let\PYG@bf=\textbf\def\PYG@tc##1{\textcolor[rgb]{0.00,0.44,0.13}{##1}}}
\expandafter\def\csname PYG@tok@se\endcsname{\let\PYG@bf=\textbf\def\PYG@tc##1{\textcolor[rgb]{0.25,0.44,0.63}{##1}}}
\expandafter\def\csname PYG@tok@sd\endcsname{\let\PYG@it=\textit\def\PYG@tc##1{\textcolor[rgb]{0.25,0.44,0.63}{##1}}}

\def\PYGZbs{\char`\\}
\def\PYGZus{\char`\_}
\def\PYGZob{\char`\{}
\def\PYGZcb{\char`\}}
\def\PYGZca{\char`\^}
\def\PYGZam{\char`\&}
\def\PYGZlt{\char`\<}
\def\PYGZgt{\char`\>}
\def\PYGZsh{\char`\#}
\def\PYGZpc{\char`\%}
\def\PYGZdl{\char`\$}
\def\PYGZhy{\char`\-}
\def\PYGZsq{\char`\'}
\def\PYGZdq{\char`\"}
\def\PYGZti{\char`\~}
% for compatibility with earlier versions
\def\PYGZat{@}
\def\PYGZlb{[}
\def\PYGZrb{]}
\makeatother

\renewcommand\PYGZsq{\textquotesingle}

\begin{document}

\maketitle
\tableofcontents
\phantomsection\label{index::doc}


Contents:


\chapter{IVR RPC 协议}
\label{rpc:ivr-rpc}\label{rpc::doc}\label{rpc:ivr}

\section{名词解释}
\label{rpc:id1}\begin{itemize}
\item {} 
IVR:internet video recorder,一种基于物联网云计算技术的网络视频录象机;
与传统NVR,DVR类似,IVR可以接入和管理各种摄像头,支持录像,支持直播点播视频等功能,
但是与传统NVR,DVR的不同在于用户可以在互联网上观看直播点播,
并且因为应用了云计算技术,IVR拥有近于无限的接入能力以及海量的并发直播点播请求;IVR主要由IVC与IVT两部分组成。

\item {} 
IVC:internet video cloud,运行于云端的视频云平台;
提供API接口以及管理网页,用户可以随时通过互联网,以网页,APP公众号等方式接入IVC,对其名下的摄像头进行管理,并进行直播点播。

\item {} 
IVT:安装与客户现场的,具有与IVC通信能力的视频设备,如摄像头,NVR等,IVT通过IVR RPC协议接入IVC,并接受IVC的管理。

\item {} 
Channel:IVT上的视频通道;
一般一路摄像头为一条channel,如果IVT为摄像头,则只有一条通道;
如果IVT为NVR,则会有多条通道,每条通道代表一个NVR管理下的摄像头。

\item {} 
IVR RPC:IVT与IVR之间采用的通信协议

\end{itemize}


\section{协议特性}
\label{rpc:id2}
IVR RPC底层基于TCP/IP的websocket,应用层为基于JSON的自定义协议。

应用层协议的基本通信模式包括RPC与event(事件通知)两种:
\begin{enumerate}
\item {} 
RPC为一应一答模式;通信中的两个端点均可以发起RPC请求,且两个方向上的RPC请求是完全独立的;但是单一方向上,只有在当前请求被应答后,才能发起下一个RPC请求

\item {} 
事件通知是没有应答的;通信中的两个端点均可以给对方发送事件通知;事件通知可以随时发起,不受当前是否有RPC正在进行的影响。

\end{enumerate}


\section{应用层协议数据包格式}
\label{rpc:id3}
此处数据包指的是应用层协议的数据包,即websocket的payload/message。数据包使用JSON标准进行串行化。


\subsection{RPC request}
\label{rpc:rpc-request}
RPC请求包的格式

\begin{Verbatim}[commandchars=\\\{\}]
\PYGZob{}
  “req”: \PYGZlt{}字符串,必填,请求的RPC方法\PYGZgt{},
  “params”: \PYGZlt{}JSON对象,可选,RPC方法的参数;当方法没有参数时,此域不存在\PYGZgt{},
  “seq”: \PYGZlt{}整数,必填,RPC的序列号;没发送一次请求,+1\PYGZgt{}
\PYGZcb{}
\end{Verbatim}


\subsection{RPC调用成功的response}
\label{rpc:rpcresponse}
当RPC调用成功,应答包的格式

\begin{Verbatim}[commandchars=\\\{\}]
\PYGZob{}
  “seq”: \PYGZlt{}整数,必填,RPC的序列号;与对应的RPC请求的序列号一致\PYGZgt{},
  “resp”: \PYGZlt{}JSON对象,必选,应答内容\PYGZgt{}
\PYGZcb{}
\end{Verbatim}


\subsection{RPC调用失败的response}
\label{rpc:id4}
当RPC调用失败,应答包的格式

\begin{Verbatim}[commandchars=\\\{\}]
\PYGZob{}
  “seq”: \PYGZlt{}整数,必填,RPC的序列号;与对应的RPC请求的序列号一致\PYGZgt{},
  “err”: \PYGZob{}
    “code”: \PYGZlt{}整数,必填,错误码\PYGZgt{},
    “msg”: \PYGZlt{}字符串,必填,错误信息\PYGZgt{}
  \PYGZcb{}
\PYGZcb{}
\end{Verbatim}


\subsection{事件通知}
\label{rpc:id5}
事件通知包的格式

\begin{Verbatim}[commandchars=\\\{\}]
\PYGZob{}
  “event”: \PYGZlt{}字符串,必填,事件名称\PYGZgt{},
  “params”: \PYGZlt{}JSON对象,可选,参数;当没有参数时,该域不存在\PYGZgt{}
\PYGZcb{}
\end{Verbatim}


\section{异常处理}
\label{rpc:id6}
当通讯的一段发现如下异常时,需主动断开websocket连接:
\begin{enumerate}
\item {} 
RPC的request与response的seq(序列号不一致)

\item {} 
在没有响应上一个RPC request的情况下收到其他RPC request

\item {} 
调用的RPC方法不存在

\item {} 
数据包格式不正确

\end{enumerate}


\section{协议流程}
\label{rpc:id7}\begin{enumerate}
\item {} 
IVT向IVC发起websocket连接,并携带上login\_code,login\_password等信息。

\item {} 
IVC测在接收到请求后,验证IVT身份,若通过,则与IVT建立websocket连接。

\item {} 
IVT定期向IVC发送keepalive事件,在keepalive事件中携带其下摄像头的状态信息。

\item {} 
根据业务安排,IVT与IVC可以进行各种RPC与event的交换。

\end{enumerate}

IVC的websocket URL格式如下:

\begin{Verbatim}[commandchars=\\\{\}]
ws://\PYGZlt{}IVC host:port\PYGZgt{}/ivc? login\PYGZus{}code=\PYGZlt{}IVT登录名\PYGZgt{}\PYGZam{}login\PYGZus{}passwd=\PYGZlt{}IVT登录密码\PYGZgt{}\PYGZam{}project=\PYGZlt{}所属项目名称\PYGZgt{}\PYGZam{}hardware\PYGZus{}model=\PYGZlt{}IVT的硬件型号\PYGZgt{}\PYGZam{}firmware\PYGZus{}model=\PYGZlt{}IVT的固件版本号\PYGZgt{}
\end{Verbatim}


\section{IVC支持的RPC方法}
\label{rpc:ivcrpc}
此处描述所有IVT可以调用的IVC的RPC方法。其中“参数”指的是RPC request数据包中的params域;
“成功应答”指的是RPC调用成功的response中的resp域;“失败应答”值得是RPC调用失败的response中的err域。


\subsection{preview\_server}
\label{rpc:preview-server}
IVT可通过该方法获取用于上传摄像头预览图的上传服务器的地址。

参数:

\begin{Verbatim}[commandchars=\\\{\}]
无
\end{Verbatim}

成功应答:

\begin{Verbatim}[commandchars=\\\{\}]
字符串,必填,上传服务器的域名/IP;如 192.168.1.120:9900
\end{Verbatim}


\subsection{ivt\_package}
\label{rpc:ivt-package}
IVT可通过该方法获取最新的固件的版本,及其下载URL。

参数:

\begin{Verbatim}[commandchars=\\\{\}]
无
\end{Verbatim}

成功应答:

\begin{Verbatim}[commandchars=\\\{\}]
\PYGZob{}
  \PYGZdq{}ivt\PYGZdq{}: \PYGZob{}
    \PYGZdq{}version\PYGZdq{}: \PYGZlt{}必填,字符串;最新固件的版本号\PYGZgt{}
    \PYGZdq{}url\PYGZdq{}: \PYGZlt{}比填,字符串;最新固件的下载地址\PYGZgt{}
    \PYGZdq{}force\PYGZdq{}: \PYGZlt{}可选,bool;是否立即升级,如果为true,则一旦没有直播,立刻升级;如果时false,则等到凌晨0\PYGZhy{}3点没有直播的时刻升级\PYGZgt{}
  \PYGZcb{}
\PYGZcb{}
\end{Verbatim}


\section{IVT支持的RPC方法}
\label{rpc:ivtrpc}
此处描述所有可以被调用的IVT的RPC方法。


\subsection{RTMPPublish}
\label{rpc:rtmppublish}
IVC可以通过该方法请求IVT publish一条RTMP流到指定URL。

参数:

\begin{Verbatim}[commandchars=\\\{\}]
\PYGZob{}
  \PYGZdq{}channel\PYGZdq{}: \PYGZlt{}必填,整数\PYGZgt{},
  \PYGZdq{}quality\PYGZdq{}: \PYGZlt{}必填,字符串;可选值为:LD、SD、HD、FHD,分别代表低清,标清,高清,全高清\PYGZgt{},
  \PYGZdq{}url\PYGZdq{}: \PYGZlt{}必填,字符串;publish RTMP流的目标URL\PYGZgt{},
  \PYGZdq{}stream\PYGZus{}Id\PYGZdq{}: \PYGZlt{}必填,字符串;用来标识这条流的ID\PYGZgt{}
\PYGZcb{}
\end{Verbatim}

成功应答(即publish成功,或该RTPM stream已经存在):

\begin{Verbatim}[commandchars=\\\{\}]
空
\end{Verbatim}


\subsection{RTMPStopPublish}
\label{rpc:rtmpstoppublish}
IVC可以通过该方法请求IVT结束正在publish的RTMP流。

参数:

\begin{Verbatim}[commandchars=\\\{\}]
\PYGZob{}
  \PYGZdq{}stream\PYGZus{}id\PYGZdq{}: \PYGZlt{}必填,字符串;RTMPPublish时给的stream\PYGZus{}id\PYGZgt{},
  \PYGZdq{}channel\PYGZdq{}: \PYGZlt{}必填,整数\PYGZgt{}
\PYGZcb{}
\end{Verbatim}

成功应答(成功结束,或该流不存在):

\begin{Verbatim}[commandchars=\\\{\}]
空
\end{Verbatim}


\section{IVC支持的event}
\label{rpc:ivcevent}
此处描述IVC接受的event事件通知。“参数”指的是事件通知数据包中的params域。


\subsection{Keepalive}
\label{rpc:keepalive}
IVT利用该事件定期向IVC通知工作状态,IVC以此作为IVT仍然在线的依据。

参数:

\begin{Verbatim}[commandchars=\\\{\}]
\PYGZob{}
  0: \PYGZlt{}可选,整数,channel 0的工作状态,如果channel 0存在的话,0为异常,1为空闲,2为直播中\PYGZgt{},
  1: \PYGZlt{}可选,整数,channel 1的工作状态,如果channel 1存在的话\PYGZgt{},
  …
\PYGZcb{}
\end{Verbatim}


\chapter{Indices and tables}
\label{index:indices-and-tables}\begin{itemize}
\item {} 
\DUspan{xref,std,std-ref}{genindex}

\item {} 
\DUspan{xref,std,std-ref}{modindex}

\item {} 
\DUspan{xref,std,std-ref}{search}

\end{itemize}



\renewcommand{\indexname}{Index}
\printindex
\end{document}
